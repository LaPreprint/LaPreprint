\section{Discussion} \label{discussion}
I have shown that \textit{Nebria brevicollis} does exhibit a circadian rhythm with predominant activity during night-time. Furthermore, I have shown that continuous light increases night-time activity levels, whereas continuous darkness decreases night-time activity levels; importantly, neither condition has any effect on the day-time activity.

Two things were surprising from our experiments: 1) Increased light intensity during night increase activity levels, and 2) that abrupt changes in light intensity cause drastic changes in activity levels, both in a nocturnal predator. 

\subsection{Increased light intensity during night-time increase activity levels} \label{light}
It is surprising to us that LL increases activity and DD decreases activity during night-time. However, a similar pattern is observed in \textit{Drosophila}, which is diurnal: If kept in DD, they become more active during  night-time \citep{Xu2021a}. This pattern suggests that interaction between light exposure and activity is complex, and an important aspect could be the discrepancy between the \textit{expected} and \textit{sensed} light intensity. One caveat is that the beetles have nowhere to hide, and the increased activity may be due to a search for shelter from the unexpected light exposure; more experiments will be needed to clarify this.

\subsection{Abrupt changes in light intensity cause drastic changes in activity levels} \label{light-change}
The second finding is the drastic changes in activity levels induced by the abrupt light-on and -off. Interestingly, although the amplitude of the change in light activity is equal for light-on and light-off, they elicit different effects in terms of their time constant. This may signify qualitative differences in the "meaning" of the signal; abrupt exposure to light could mean exposure to a predator, whereas darkness means a chance to hunt. \par
While there is an abundance of work on exposure to different light conditions over periods of 24 hours in the circadian rhythm literature, I have yet to discover a single paper that acknowledges the drastic changes around light-on and light-off, or uses a more realistic, gradual increase in light intensity. In the behavioural ecology literature there are a few hints that the gradual change is important. \cite{Kelber2006} found that light intensity plays a crucial role in the foraging activity of nocturnal and crepuscular solitary bees, and only at dawn and dusk will they emerge. \cite{Nielsen1962} shows that mosquitoes will start swarming only if light intensity increases gradually; if it goes to maximum abruptly they fail to swarm altogether. So while there are indications in the literature that the way in which the stimulus is changed is of importance, it has yet to translate into the laboratories studying sleep and circadian rhythms. One can imagine the different signals light intensity changes can deliver; where a slow, gradual increase might signal a sunrise, an abrupt increase can mean exposure to predators, and thus evoke an escape response. \par
Additionally, when testing the sensory threshold (one of the main criteria of sleep), great care should be taken to not use stimuli which may influence the circadian rhythm, as the observed effects on sleep may be indirectly mediated by circadian processes. It also emphasises the importance of using sleep posture to classify sleep; something which is currently used as a criterion, but mostly ignored in present experimental protocols.


\subsection{Focus on sleep-wake transitions and the need of a better model of sleep} \label{model}
I am still trying to concretise my thoughts and ideas relating various sleep processes, sensory processing and circadian rhythms, as these phenomena are harder to dissect than one would think.
As the vast majority of animals exhibit circadian rhythms, any model which aims to predict sleep/wake state will almost inevitably be dominated by this rhythm. And indeed, the prevailing model of sleep, introduced by \cite{Borbely2016} is based on two slow components, the circadian rhythm and homeostatic regulation. However, this model completely fails to explain everyday phenomena such as waking up during the night or napping. Whereas humans have engineered our homes to accommodate consolidated periods of sleep, animals live a much less sheltered environments where disturbances are frequent and need to be taken seriously; thus the circadian rhythm may be seen not to deliver long consolidated periods of sleep, but rather to cluster frequent shorter bouts of sleep. These brief, fast events are not less valid, and a thorough model of sleep should be able to explain them. A recent approach is to predict transitions between sleep/wake states rather than the states themselves. \cite{Wiggin2020a} predicts the probability of \textit{Drosophila} switching state based on an animals immediate history of activity. \cite{Ni2019a} approaches the same problem from a more conceptual point of view, and not only describes the homeostatic component but also the effect of arousal, a faster component. They show how these fast and slow processes are integrated by neurons in the dorsal fan-shaped body (dFB), and thus provides physiological evidence for the influence of fast processes in regulating sleep.

It is thus evident that a new conceptual framework, a model, is needed to better account for recent insights derived from behavioural and physiological work. I've started thinking about what such a model should entail, and what Marr-esque levels it should aim at. I am thinking about it as an ecological model, which should be able to take into account the relevant sensory stimuli as well as their implicit signals. I am however confident that processes that influence whether an animal dozes or wakes can broadly be divided into two categories: Slow- and fast-acting. 
A sufficient model should be able to account for the following scenarios:
\begin{itemize}
    \item What is the homeostatic component? (sensory/motor systems, processing?)
    \item Non-consolidated sleep bouts
    \item The evolution of sleep-loss in cave-fish \citep{Duboue2011}
\end{itemize}


% \sidenote{What is the definition of arousal in \textit{Drosophila} work?}

% \subsubsection{Model of sleep processes, environmental stimuli and physiological mechanisms}
% A model is needed.

% \subsubsection{Arousal is not just arousal}

% \subsubsection{Stimuli are not just stimuli}

% \subsection{A note on the sensory threshold criterion}

% \subsection{Pressures influencing dozing and waking}


% Falling asleep is primarily the result of homeostatic pressure, though it may happen \textit{in spite of} ecologically relevant stimuli.
% Waking up, is rather opposite, waking up wether because the homeostatic pressure is no longer sufficient for the animal to remain asleep, but it may also be that environmentally relevant stimuli, such as threats or the smell of food can trigger waking.

% \subsubsection{Slow processes}
% \cite{Borbely2016} introduced a two-process model, containing two opposing slow processes; a homeostatic process (Process S) and a circadian pacemaker (Process C)\sidenote{Put figure of pressures here}. However, slow processes alone cannot account for all sleep/wake transitions. In humans, sleep disturbances are events where factors such as environmental noise causes one to wake up. Clearly, such factors act over very short periods of time. Interestingly, a key condition in determining whether an animal sleeps is a change in the sensory threshold. This is determined in sensory threshold experiments, where immediate factors such as light, temperature or mechanosensory stimuli are used to wake the animal up. 

% \subsubsection{Fast processes}

% Here I take a neuroecological approach to the transitions between sleep and wakefulness; waking and dozing. Rather than assume that all wakenings and dozes are similar, we consider the possible reasons for the animal to transition. 
% Importantly, they showed that it is homeostatically regulated, and thus that there is a internal homeostatic pressure. However, as homeostatic regulation of sleep is acting on time-scales of hours, it cannot account for the brief awakenings. 
% From work on circadian rhythms, it has been shown that both light and temperature are important environmental factors maintaining a circadian rhythm in ground beetles \citep{Constantinou1985}.

% It prompts the question: What does an abrupt environmental change \textit{mean} to the animal. It depends. A sudden change from darkness to light means exposure, and is a high risk situation that needs to be dealt with quickly whether a predator is present or not. A change in the opposite direction, from light to darkness, could either mean improved cover or the looming of a predator, and in the face of ambiguity the best response might be stay put. 


% An animal that can detect a predator has the capability to integrate information about a potential threat from various sensory modalities: The sensory stimuli are integrated and \textit{perceived} as a potential threat, causing an escape response.

% In a beautiful study \cite{French2021} showed that \textit{Drosophila} processes sensory information during sleep, and that stimuli with different saliencies elicit different responses. 

% \textbf{MODEL}
% \begin{center}
%     \begin{tabularx}{\textwidth}{c|X|X|}
%     & Internal & External \\
%     \hline
%     Fast & Stochastic neural processes & Rate of change of light intensity, temperature\\
%     \hline
%     Slow & Circadian rhythm, homeostatic process & Ambient light intensity, temperature \\
%     \hline
%     % Ultra slow & Ultradian rhythm & \\
%     % \hline
%     \end{tabularx}
%     \smallskip
%     \captionof{table}{Processes influencing sleep transitions.} \label{tab:model-processes} 
% \end{center}

% % As a rule, fast changes are more important. Abrupt changes in light intensity or temperature could mean that a predator is very near.




% From an ecological perspective, it is intriguing to consider what the stimulus "means" to the beetle; a normal sunrise or sunset does not happen abruptly, but rather gradually. An abrupt change in perceived light intensity could well mean danger for the animal, and thus the immediate activity could be interpreted as an escape response. As my project centers on the transitions between sleep and wakefulness, it is thus useful to consider the types of stimuli used to wake animals, what they mean to the animal. Recording local field potentials from a brain of an animal that goes directly to an escape response might well be radically different from that coming from a gradual change in  light intensity - or an appetitive odour of vinegar.

% A picture emerges, where waking is not considered a simple single event. Rather, it is an ensemble of processes, qualitatively different in valence and risk, with a single behavioural outcome: Wakefulness. 

% Thus we are bridging neuroscience and sensory- and behavioural ecology, and a neuroecological study emerges. 

% We have started recording preliminary electroretinograms (ERG) to get an idea of how the eye adjusts to light.

% \textbf{How does the circadian system influence achieve/influence sleep?} A central circadian clock can modulate the sensitivity at various levels, both centrally and peripherally. Conceptually, if the modulation happens at the primary sensor level, do we still call it sleep? It is at least a different way of achieving sleep - turning down sensors peripherally. \textbf{We need to ensure that the peripheral sensor we are trying to stimulate is not itself under circadian control, or adjust for it}. The retina undergoes circadian rhythm, can we then infer anything from a sensory threshold test? \cite{Yoshii2005}
% Most work focuses on photoperiod. However, in ground beetles, the sensitivity of retina is itself modulated by the circadian rhythm \cite{Fleissner1982, Koehler1978}. 
% The photic entrainment of the circadian rhythm may be carried out only by a separate opsin, showing wavelength selectivity (in crickets, \citealt{Komada2015}).

% Sleep \textit{is} not a circadian rhythm. The probability of \text{being asleep} is affected by the circadian rhythm/clock. Circadian rhythms are physiological patterns that changes rhythmically during a 24 hour period in the absence of external sensory input. Beetle ERGs show that the sensitivity of the retina express a circadian rhythm. That has profound implications to our definition of sleep. The best estimate we have of sleep is that the sensory threshold to a sensory stimulus is elevated; however, this is thought to reflect a central process of sleep, but may in fact just be due to a change in peripheral sensitivity e.g. in the retina.

\subsection{Outlook} \label{outlook}
Looking ahead, I can see a clear path, and questions that will be important to answer, beyond the gradual light experiments about to take place (preliminary timeline is found in Appendix \ref{fig:timeline}).

\begin{itemize}
    \item Better understanding of abiotic factors in the natural environment of \textit{Nebria brevicollis}
    \item Establish sleep posture and sensory threshold
    \item Measure the peripheral and central effects of light on sleep/activity
\end{itemize}

\subsubsection{Better understanding of abiotic factors in the natural environment of \textit{Nebria brevicollis}}
It is evident that understanding the ecology of your experimental organism is important. Since little is documented about the abiotic factors of the habitat of \textit{Nebria brevicollis}, this is data which needs to be collected. We will need to measure light intensity, temperature, and potentially soil humidity under the leaf litter in our temperate forest during 24 hours at different points during the year.

\subsubsection{Establish sleep posture and sensory threshold}
The next immediate step is to record beetles on a trackball, where detailed postural observations can be made. This only requires a small amount of work to get started. I need to finish Bux, and finish my design of the trackball. Additionally, building an odour stimulator will be needed for sensory threshold experiments, but this equipment has already been developed by André Maia de Chagas, and we will begin the work during June.

\subsubsection{Dissociate the peripheral and central effects of light}
Lastly, the electrophysiological experiments. These experiments would entail recording from the retina (ERG) and the central complex (LFP) simultaneously, to disentangle the effects of the circadian clock on sleep directly and indirectly by modulation of peripheral photosensation. Everything has been purchased, and the rig now just needs to be set up. An ERG setup was recently built in our lab for another beetle species so starting these experiments should be feasible in the near future.
