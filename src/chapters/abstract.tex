\begin{abstract}
    The headline of year 2 was a change of model organism; we decided to move from wood ants to the ground beetle \textit{Nebria brevicollis}, a nocturnal predatory beetle. After a few months of adjustments, experiments are now running regularly. This work sought to investigate activity patterns and their relationship with light conditions during a 24 hour period. We had three experimental light conditions: 12h:12h light and darkness (LD), constant light (LL) and constant darkness (DD). We showed that \textit{Nebria brevicollis} is predominantly active during night-time, also in under experimental conditions. Furthermore, we found that lighting conditions affect night-time but not daytime activity; constant light increased night-time activity whereas constant darkness decreased night-time activity. These results suggest that \textit{Nebria brevicollis} are either insensitive to varying ambient light levels during daytime, or that light levels do not carry ecologically important information during daytime. This work has helped us get closer to a description of potential sleep in a ground beetle.
\end{abstract}