\section{Introduction} \label{intro}
It is widely accepted that circadian processes influence sleep. Therefore gaining a better understanding of the interaction between circadian processes and sensory stimuli is a natural first step into understanding sleep. Here we set out to explore the effects of different light regimes upon the activity levels of a nocturnal predatory ground beetle, \textit{Nebria brevicollis}.

\subsection{The ground beetle \textit{Nebria brevicollis}}
\textit{Nebria brevicollis} is a nocturnal ground beetle found throughout the Palearctic, and abundant in temperate forests in the United Kingdom (Figure \ref{fig:forest}). These beetles are between 1.5 and 2cm long. They breed during fall and emerge during spring. The phenology\sidenote{"\textbf{Phenology} is the study of periodic events in biological life cycles and how these are influenced by seasonal and interannual variations in climate, as well as habitat factors (such as elevation)." Wikipedia.} of \textit{Nebria brevicollis} is dependent on the  animals' habitat. \cite{Williams1959} found that beetles caught in woodland show activity peaks during autumn, whereas beetles caught in grassland show activity peaks during spring. Both populations are drastically less active during July and August, and it has been suggested that they are an aestivating\sidenote{\textbf{Aestivation} is the equivalent of hibernating during summer.} species \citep{Pozsgai2018}. 

\subsection{Pathways mediating photoentrainment}
As most animals, insects show circadian rhythms of activity classified as diurnal, nocturnal or crepuscular, depending on their peak of activity. These rhythms are entrained by both light and temperature (photoentrainment and thermoentrainment) with the relative importance of each being species-dependent. As light levels and temperature are mostly coupled in nature, we rely on laboratory experiments to distinguish between their effects. The influence of light on the circadian clock has received much attention across a wide array of insects (reviewed in \citealt{Helfrich-Forster2020}). In \textit{Drosophila} the predominant circadian entrainment is mediated by photoreceptors and the circadian rhythm can be abolished altogether if all photoreceptors are eliminated \citep{Helfrich-Forster2001}. 
One of the main mechanisms in \textit{Drosophila} relies on cryptochrome (CRY), a photopigment present in all the clock cells within the brain. However, most other insects have thicker cuticle and lack photosensitive CRY. 
\cite{Komada2015} found that green-sensitive opsin is responsible for photoperiodic entrainment of circadian rhythm in the cricket \textit{Gryllus bimaculatus}. However, insect species differ in their spectral sensitivities, and their findings may thus differ depending on the species, but it raises the important point that only a subset of wavelengths may contribute to the entrainment of circadian rhythm.
Interestingly, the retinal sensitivity has been found to be under circadian regulation in cockroaches, praying mantids and various species of beetles, significantly decreasing during the day as recorded by electroretinograms (ERG) \citep{Wills1985,Schirmer2014,Fleissner1982,Koehler1978}. Surgical removal of ocelli in cockroaches does not abolish the entrainment, whereas painting of the compound eyes does, suggesting that the compound eyes are responsible for photoentrainment \citep{Roberts1965}.

\subsection{Interaction between temperature and activity}
Though the present work focuses on the impact of light, a note has to be made on temperature. Temperature is actively used to guide behaviour in insects. In \textit{Drosophila}, temperature entrains circadian rhythm which is otherwise abolished by constant light \citep{Yoshii2005}, called thermoentrainment. They show temperature preferences that vary over the course of a day \citep{Kaneko2012}, and their activity levels can be modulated by brief periods of temperature changes \citep{Alpert2020}. In many other insects, thermoentrainment of the circadian rhythm is even more important; temperature and light intensity signals can interact and integrate to produce extreme circadian phenotypes, as seen in beetles and linden bugs \citep{Constantinou1985,Kaniewska2020}. Lastly, this becomes all the more important as the mechanisms for temperature sensation differs drastically between \textit{Drosophila} and most other insects, although it seems true for all insects that thermosensation is centred around the antennae (reviewed in \citealt{Gonzalez-Tokman2020a}).






